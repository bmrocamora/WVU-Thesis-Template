%% USPSC-fichacatalografica.tex
% ---
% Inserir a ficha bibliografica
% ---
% Isto é um exemplo de Ficha Catalográfica, ou ``Dados internacionais de
% catalogação-na-publicação''. Você pode utilizar este modelo como referência. 
% Porém, provavelmente a biblioteca da sua universidade lhe fornecerá um PDF
% com a ficha catalográfica definitiva após a defesa do trabalho. Quando estiver
% com o documento, salve-o como PDF no diretório do seu projeto e substitua todo
% o conteúdo de implementação deste arquivo pelo comando abaixo:
%
\begin{fichacatalografica}
	\hspace{-1.4cm}
	\imprimirnotaautorizacao \\ \\
	%\sffamily
	\vspace*{\fill}					% Posição vertical
\begin{center}					% Minipage Centralizado
  \imprimirnotabib \\
  \begin{table}[htb]
	\scriptsize
	\centering	
	\begin{tabular}{|p{0.9cm} p{8.7cm}|}
		\hline
	      & \\
		  &	  \imprimirautorficha     \\
		
		 \imprimircutter & 
							\hspace{0.4cm}\imprimirtitulo~  / ~\imprimirautor~ ;  ~\imprimirorientadorcorpoficha. -- 	\imprimirlocal, \imprimirdata.   \\
		
		  &  % Para incluir nota referente à versão corrigida no corpo da ficha,
			  % incluir % no início da linha acima e tirar a % do início da linha abaixo
			  %	\hspace{0.4cm} \imprimirtitulo~  / ~\imprimirautor~ ; ~\imprimirorientadorcorpoficha~- ~\imprimirnotafolharosto. -- \imprimirlocal, \imprimirdata.  \\
		
			\hspace{0.4cm}\pageref{LastPage} p. : il. (algumas color.) ; 30 cm.\\ 
		  & \\
		  & 
		    \hspace{0.4cm}\imprimirnotaficha ~--~ 
						  \imprimirunidademin, 
						  \imprimiruniversidademin, 
		                  \imprimirdata. \\ 
		  & \\                 
		   % Para incluir nota referente à versão corrigida em notas,
		    % incluir uma % no início da linha acima e	
		    % tirar a % do início da linha abaixo
		    % & \hspace{0.4cm}\imprimirnotafolharosto \\ 
		  & \\ 
		  & \hspace{0.4cm}1. LaTeX. 2. abnTeX. 3. Classe USPSC. 4. Editoração de texto. 5. Normalização da documentação. 6. Tese. 7. Dissertação. 8. Documentos (elaboração). 9. Documentos eletrônicos. I. \imprimirorientadorficha. 
		   II. Título. \\
	
		     %Se houver co-orientador, inclua % antes da linha (antes de II. Título.) 
		     %          e tire a % antes do comando abaixo 
		     %III. Título. \\   
		  \hline
	\end{tabular}
  \end{table}
\end{center}
\end{fichacatalografica}
% ---


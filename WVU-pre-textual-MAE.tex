%% WVU-pre-textual-MAE.tex
%% Camandos para definição do tipo de documento (tese ou dissertação), área de concentração, opção, preâmbulo, titulação 
%% referentes aos Programas de Pós-Graduação
\instituicao{West Virginia University}
\unidade{BENJAMIN M. STATLER COLLEGE OF ENGINEERING AND MINERAL SCIENCES}
\unidademin{Benjamin M. Statler College of Engineering and Mineral Sciences}
\universidademin{West Virginia University}

% A EESC não inclui a nota "Versão original", portanto o comando abaixo não tem a mensagem entre {}
\notafolharosto{ }
%Para a versão corrigida tire a % do comando/declaração abaixo e inclua uma % antes do comando acima
%\notafolharosto{VERS\~AO CORRIGIDA}
% ---
% dados complementares para CAPA e FOLHA DE ROSTO
% ---
\universidade{WEST VIRGINIA UNIVERSITY}
\titulo{\LaTeX Thesis Template using WVU package for MAE}
\titleabstract{\LaTeX Thesis Template using WVU package for MAE}
\tituloresumo{\LaTeX Thesis Template using WVU package for MAE}
\autor{Bernardo Martinez Rocamora}
\autorficha{Martinez Rocamora, Bernardo}
\autorabr{MARTINEZ ROCAMORA, B.}

\cutter{S856m}
% Para gerar a ficha catalográfica sem o Código Cutter, basta 
% incluir uma % na linha acima e tirar a % da linha abaixo
%\cutter{ }

\local{Morgantown, WV}
\data{2022}
% Quando for Orientador, basta incluir uma % antes do comando abaixo
\renewcommand{\orientadorname}{Advisor:}
% Quando for Coorientadora, basta tirar a % utilizar o comando abaixo
%\newcommand{\coorientadorname}{Coorientador:}
\orientador{Prof. Dr. Guilherme Pereira}
\orientadorcorpoficha{advisor Guilherme Pereira}
\orientadorficha{Pereira, Guilherme, adv.}
%Se houver co-orientador, inclua % antes das duas linhas (antes dos comandos \orientadorcorpoficha e \orientadorficha) 
%          e tire a % antes dos 3 comandos abaixo
%\coorientador{Prof. Dr. Jo\~ao Alves Serqueira}
%\orientadorcorpoficha{orientadora Elisa Gon\c{c}alves Rodrigues ;  co-orientador Jo\~ao Alves Serqueira}
%\orientadorficha{Rodrigues, Elisa Gon\c{c}alves, orient. II. Serqueira, Jo\~ao Alves, co-orient}

\notaautorizacao{I AUTHORIZE THE PARTIAL AND TOTAL REPRODUCTION, BY CONVENTIONAL OR ELECTRONIC MEANS, FOR PURPOSES OF EDUCATION AND RESEARCH, GIVEN THE APPROPRIATE CREDITS.}
\notabib{~  ~}

\newcommand{\programa}[1]{

% DAERO ==========================================================================
\ifthenelse{\equal{#1}{DAERO}}{
    \tipotrabalho{Tese (Doutorado)}
    \tipotrabalhoabs{Thesis (Doctorate)}
    \area{Aerospace Engineering}
	%\opcao{Nome da Opção}
    % O preambulo deve conter o tipo do trabalho, o objetivo, 
	% o nome da instituição, a área de concentração e opção quando houver
	\preambulo{Thesis presented to the Benjamin M. Statler College of Engineering and Mineral Sciences at West Virginia University for the requirements of the title of Doctor of Philosophy of the Aerospace Engineering Graduate Program.}
	\notaficha{Thesis (Doctorate) - Aerospace Engineering Graduate Program}
    }{
% MAERO ===========================================================================
\ifthenelse{\equal{#1}{MAERO}}{
	\tipotrabalho{Disserta\c{c}\~ao (Mestrado)}
	\tipotrabalhoabs{Thesis (Master)}
	\area{Aerospace Engineering}
	%\opcao{Nome da Opção}
	% O preambulo deve conter o tipo do trabalho, o objetivo, 
	% o nome da instituição, a área de concentração e opção quando houver
	\preambulo{Thesis presented to the Benjamin M. Statler College of Engineering and Mineral Sciences at West Virginia University for the requirements of the title of Doctor of Philosophy of the Aerospace Engineering Graduate Program.}
	\notaficha{Thesis (Doctorate) - Aerospace Engineering Graduate Program}}
    }{
% DMECH =======================================================================
\ifthenelse{\equal{#1}{DMECH}}{
    \tipotrabalho{Tese (Doutorado)}
    \tipotrabalhoabs{Thesis (Doctor)}
    \area{Mechanical Engineering}
	%\opcao{Nome da Opção}
    % O preambulo deve conter o tipo do trabalho, o objetivo, 
	% o nome da instituição, a área de concentração e opção quando houver
	\preambulo{Thesis presented to the Benjamin M. Statler College of Engineering and Mineral Sciences at West Virginia University for the requirements of the title of Doctor of Philosophy of the Aerospace Engineering Graduate Program.}
	\notaficha{Thesis (Doctorate) - Aerospace Engineering Graduate Program}}
    }{
% MMECH ===========================================================================
\ifthenelse{\equal{#1}{MMECH}}{
	\tipotrabalho{Disserta\c{c}\~ao (Mestrado)}
	\tipotrabalhoabs{Thesis (Master)}
	\area{Mechanical Engineering}
	%\opcao{Nome da Opção}
	% O preambulo deve conter o tipo do trabalho, o objetivo, 
	% o nome da instituição, a área de concentração e opção quando houver
	\preambulo{Thesis presented to the Benjamin M. Statler College of Engineering and Mineral Sciences at West Virginia University for the requirements of the title of Master of Science of the Mechanical Engineering Graduate Program.}
	\notaficha{Thesis (Doctorate) - Aerospace Engineering Graduate Program}}
    }{
% Others
	\tipotrabalho{Disserta\c{c}\~ao/Tese (Mestrado/Doutorado)}
	\tipotrabalhoabs{Dissertation/Thesis (Master/Doctor)}
    \area{Nome da \'Area}
    \opcao{Nome da Op\c{c}\~ao}
    % O preambulo deve conter o tipo do trabalho, o objetivo, 
	% o nome da instituição, a área de concentração e opção quando houver
	\preambulo{Disserta\c{c}\~ao/Tese apresentada \`a Escola de Engenharia de S\~ao Carlos da Universidade de S\~ao Paulo, para obten\c{c}\~ao do t\'itulo de Mestre/Doutor em Ci\^encias - Programa de P\'os-Gradua\c{c}\~ao em Engenharia.}
	\notaficha{Disserta\c{c}\~ao/Tese (Mestrado/Doutorado) - Programa de P\'os-Gradua\c{c}\~ao e \'Area de Concentra\c{c}\~ao em Engenharia}		
    }}
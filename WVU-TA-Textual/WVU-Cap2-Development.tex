%% WVU-Cap2-Development.tex 

% ---
% Este capítulo, utilizado por diferentes exemplos do abnTeX2, ilustra o uso de
% comandos do abnTeX2 e de LaTeX.
% ---

\chapter{Development}\label{ch_development}
This chapter is the main part of the scholarly work and should contain an orderly and detailed exposition of the subject. It is divided into sections and subsections, in accordance with the approach to the theme and method, covering: literature review, materials and methods, techniques used, results obtained and discussion.

Below are minimal examples of tables, charts, document divisions and other items.

\section{Resultados de comandos}\label{sec-divisoes}

% ---
\subsection{Tables and charts}

O \textbf{Tutorial do Pacote USPSC para modelos de trabalhos de acad\^emicos em LaTeX - vers\~ao 3.1} apresenta orientações completas e diversas formatações de tabelas, dentre elas a \autoref{tab-ibge}, que é um exemplo de tabela alinhada que pode ser longa ou curta, conforme padrão do Instituto Brasileiro de Geografia e Estatística (IBGE).

%\begin{table}[H]
\begin{table}[htb]
	\IBGEtab{%
		\caption{Frequência anual por categoria de usuários}%
		\label{tab-ibge}
	}{%
		\begin{tabular}{ccc}
			\toprule
			Categoria de Usuários & Frequência de Usuários \\
			\midrule \midrule
			Graduação & 72\% \\
			\midrule 
			Pós-Graduação & 15\% \\
			\midrule 
			Docente & 10\% \\
			\midrule 
			Outras & 3\% \\
			\bottomrule
		\end{tabular}%
	}{%
		\fonte{Elaborada pelos autores.}%
		\nota{Exemplo de uma nota.}%
		\nota[Anotações]{Uma anotação adicional, que pode ser seguida de várias
			outras.}%
		
	}
\end{table}


The formatting of the chart is similar to the table, but it must have its sides closed and contain the horizontal lines.
\newpage

% o comando \newpage foi utilizado para forçar a quebra de página

\begin{quadro}[htb]
	\caption{\label{quadro_modelo}Níveis de investigação}
	\begin{tabular}{|p{2.6cm}|p{6.0cm}|p{2.25cm}|p{3.40cm}|}
		\hline
		\textbf{Nível de Investigação} & \textbf{Insumos}  & \textbf{Sistemas de Investigação}  & \textbf{Produtos}  \\
		\hline
		Meta-nível & Filosofia\index{filosofia} da Ciência  & Epistemologia &
		Paradigma  \\
		\hline
		Nível do objeto & Paradigmas do metanível e evidências do nível inferior &
		Ciência  & Teorias e modelos \\
		\hline
		Nível inferior & Modelos e métodos do nível do objeto e problemas do nível inferior & Prática & Solução de problemas  \\
		\hline
	\end{tabular}
	\begin{flushleft}
		%\fonte{\citeonline{van1986}}
		Fonte: \citeonline{van1986}
	\end{flushleft}
\end{quadro}

% ---
\subsection{Figures}\label{sec_figuras}
% ---
\index{figuras} Figures can be created directly in \LaTeX, like the example \autoref{fig_circulo}. \\ 

\begin{figure}[htb]
	\caption{\label{fig_circulo}A delimitação do espaço}
	\begin{center}
		\setlength{\unitlength}{9cm}
		\begin{picture}(1,1)
		\put(0,0){\line(0,1){1}}
		\put(0,0){\line(1,0){1}}
		\put(0,0){\line(1,1){1}}
		\put(0,0){\line(1,2){.5}}
		\put(0,0){\line(1,3){.3333}}
		\put(0,0){\line(1,4){.25}}
		\put(0,0){\line(1,5){.2}}
		\put(0,0){\line(1,6){.1667}}
		\put(0,0){\line(2,1){1}}
		\put(0,0){\line(2,3){.6667}}
		\put(0,0){\line(2,5){.4}}
		\put(0,0){\line(3,1){1}}
		\put(0,0){\line(3,2){1}}
		\put(0,0){\line(3,4){.75}}
		\put(0,0){\line(3,5){.6}}
		\put(0,0){\line(4,1){1}}
		\put(0,0){\line(4,3){1}}
		\put(0,0){\line(4,5){.8}}
		\put(0,0){\line(5,1){1}}
		\put(0,0){\line(5,2){1}}
		\put(0,0){\line(5,3){1}}
		\put(0,0){\line(5,4){1}}
		\put(0,0){\line(5,6){.8333}}
		\put(0,0){\line(6,1){1}}
		\put(0,0){\line(6,5){1}}
		\end{picture}
	\end{center}
	\legend{Fonte: \citeonline{equipeabntex2}}
\end{figure}

% ---
\section{Document hierarchy}\label{sec-hierarchy-b}

This section exemplifies the use of document divisions in accordance with ABNT NBR 6024 \cite{nbr6024}.
% ---
% ---
\subsection{Document hierarchy: subsection}\label{sec-hierarchy-subsection}
% ---

An example section is \autoref{sec-hierarchy-b}. This is the \autoref{sec-divisions-subsection}.

\subsubsection{Document hierarchy: subsubsection}\label{sec-hierarchy-subsubsection}

This is a \texttt{subsubsection} of \LaTeX.

\subsubsection{Document hierarchy: subsubsection}

This is another sub-sub-section.

\subsection{Document hierarchy: subsection}\label{sec-example-subsec}

This is another sub-section.

\subsubsection{Document hierarchy: subsubsection}

This is yet another sub-section of \autoref{sec-example-subsec}.

\subsubsubsection{This is the fifth}\label{sec-example-subsubsubsection}

This is a fifth level section. It is produced with the following command:

\begin{verbatim}
\subsubsubsection{Esta é uma subseção de quinto
nível}\label{sec-example-subsubsubsection}
\end{verbatim}

\subsubsubsection{Esta é outra subseção de quinto nível}\label{sec-example-subsubsubsection-outro}

Esta é outra seção de quinto nível.


\paragraph{This is a numbered paragraph}\label{sec-example-paragrafo}

This is an example numbered paragraph. It is produced with the paragraph command:

\begin{verbatim}
\paragraph{This is a numbered paragraph}\label{sec-example-paragrafo}
\end{verbatim}

The numbering between numbered paragraphs and sub-sub-subsections is continuous.

\paragraph{This is another numbered paragraph.}\label{sec-example-paragrafo-outro}

This is yet another numbered paragraph.

% ---
\subsection{This is an example of a long subsection name that applies to sections and other document hierarchy. It must be left-aligned and the second and other lines must start just below the first word of the first line} 

Note that heading alignment follows this rule in the table of contents as well.	






